\section{Desarrollo Experimental}
	\setlength{\parindent}{1em}
	\setlength{\parskip}{10pt}
	\subsection{Objetivo Espec\'ifico}
	\begin{enumerate}
		\item Desarrollar en la FPGA un programa en VHDL de una ALU,  en cu\'al se puedan seleccionar tres tipos de operaciones, L\'ogicas, Aritm\'eticas y Shifters, est\'as son:
		\begin{itemize}
			\item L\'ogicas (A y B de 10 bits):
			\begin{enumerate}
				\item Negaci\'on o Complemento a 1 de A.
				\item Complemento a 2 de A.
				\item AND entre A y B.
				\item OR entre A y B.
			\end{enumerate}
			\item Shifters (A de 10 bits):
			\begin{enumerate}
				\item LSL (Logical Shift Left).
				\item ASR. (Arithmetic Shift Right).
			\end{enumerate}
			\item Aritm\'eticas (A y B)
			\begin{enumerate}
				\item Suma 1 byte c/ carry out.
				\item Resta 1 byte.
				\item Multiplicaci\'on de 5 bits
			\end{enumerate}
		\end{itemize}
		\item Con los siguientes especificaciones:
		\begin{itemize}
			\item Realizar el programa en VHDL que incluya todas las operaciones b\'asicas (L\'ogicas, Shifters, Aritm\'eticas ) del tama\~no de dato correcto como se solicita en la descripci\'on.
			\item Recuerde que todas las operaciones anteriores deben funcionar como una unidad, para este inciso se deben programar todas las opciones por componentes o multi procesos.
			\item Para las opciones de corrimientos de debe incluir un clock para sincronizar los desplazamientos.
			\item Todas las entradas deben ser consideradas dependiendo de la operaci\'on a realizar, estas deber\'an ser introducidas por medio de microswitches.
			\item La asignaci\'on de pines se debe de hacer de acuerdo a la disponibilidad en las terminales de salida en la FPGA.
			\item La salida del bloque aritm\'etico se tendr\'a que ser desplegada en el display de 7 seg. 4 digitos y a su vez la salida de los shifters y la unidad l\'ogica se  mostrara \'unicamente mediante Leds.
		\end{itemize}
		\item Se manejara el mismo programa del inciso anterior pero ahora todas las operaciones tanto Aritm\'eticas,  shifters  o  l\'ogicas se  mostraran mediante mensaje  de  texto  pregrabado  en memoria ROM indexada en VHDL para el texto identificador de la operaci\'on por ejemplo: SunnA,  rEstA,  nnulti,  And, or,  not,  Co1,  Co2,  ror,  roL,  LSL,LSren  tipo  ventana  deslizante display de 4 d\'igitos de tipo \'anodo com\'un de 7 seg. El cual se multiplexar\'a para que se despliegue 5 segundos antes de mostrar el resultado de la operaci\'on. Cada  una  de  las  letras  ser\'a  almacenada  en  un  solo  byte,  y  se  multiplexar\'a similar  a  los n\'umeros pero con ventana deslizante que significa que el mensaje se desplaza para mostrar todas las letras durante 5 segundos de forma autom\'atica antes de mostrar el resultado.
	\end{enumerate}
\clearpage